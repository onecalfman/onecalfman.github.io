\documentclass[]{article}

\usepackage{amsmath, amssymb, amsthm}
\usepackage[T1]{fontenc}
\usepackage{mathptmx}
\usepackage{multicol}
\usepackage[utf8]{inputenc}
\usepackage{geometry}
 \geometry{
	 a4paper,
	 left=10mm,
	 right=10mm,
	 top=2mm,
	 bottom=2mm,
	 }
\usepackage{hyperref}
\title{Mengen}
\begin{document}

\maketitle

\paragraph{Beispiel:}
\texttt{\href{onecalfman.github.io/mengen?set=finger\&t=500\&s=5\&p=1\&r=10}{onecalfman.github.io/mengen?set=finger\&t=500\&s=5\&p=1\&r=10}} \\
\begin{multicols}{2}

	\paragraph{set}  Legt die verwendeten Bilder fest
	\begin{itemize}
		\item Default: augen
		\item Möglich: augen,finger,punkte
	\end{itemize}

	\paragraph{t} Anzeigezeit in Millisekunden

	\begin{itemize}
		\item Default: 7
		\item Range: $ 100 - 1500 \in \mathbb{N}$
	\end{itemize}

	\paragraph{s} Verkürzung der Anzeigezeit pro Runde in Prozent. Zum Beispiel wird für s=2 die Anzeigezeit pro runde um 2\% verringert.
	\begin{itemize}
		\item Default: 0
		\item Range: $ 1 - 10 \in \mathbb{N}$
	\end{itemize}

	\paragraph{r}  Runden pro Durchgang
	\begin{itemize}
		\item Default: 20
		\item Range: $ 5 - 30 \in \mathbb{N}$
	\end{itemize}

	\paragraph{c}  Wenn c gesetzt wird werden keine Hintergrundfarben für die Buttons angezeigt.
	\begin{itemize}
		\item Default: An
		\item Range: egal (ASCII-Zeichen)
	\end{itemize}

	\paragraph{p}  Wenn p gesetzt wird, wird das Endergebnis in Prozent, statt als x von y angezeigt.
	\begin{itemize}
		\item Default: x von y
		\item Range: egal (ASCII-Zeichen)
	\end{itemize}

	\end{multicols}


\end{document}
