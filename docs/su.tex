%%
% Copyright (c) 2017 - 2020, Pascal Wagler;
% Copyright (c) 2014 - 2020, John MacFarlane
%
% All rights reserved.
%
% Redistribution and use in source and binary forms, with or without
% modification, are permitted provided that the following conditions
% are met:
%
% - Redistributions of source code must retain the above copyright
% notice, this list of conditions and the following disclaimer.
%
% - Redistributions in binary form must reproduce the above copyright
% notice, this list of conditions and the following disclaimer in the
% documentation and/or other materials provided with the distribution.
%
% - Neither the name of John MacFarlane nor the names of other
% contributors may be used to endorse or promote products derived
% from this software without specific prior written permission.
%
% THIS SOFTWARE IS PROVIDED BY THE COPYRIGHT HOLDERS AND CONTRIBUTORS
% "AS IS" AND ANY EXPRESS OR IMPLIED WARRANTIES, INCLUDING, BUT NOT
% LIMITED TO, THE IMPLIED WARRANTIES OF MERCHANTABILITY AND FITNESS
% FOR A PARTICULAR PURPOSE ARE DISCLAIMED. IN NO EVENT SHALL THE
% COPYRIGHT OWNER OR CONTRIBUTORS BE LIABLE FOR ANY DIRECT, INDIRECT,
% INCIDENTAL, SPECIAL, EXEMPLARY, OR CONSEQUENTIAL DAMAGES (INCLUDING,
% BUT NOT LIMITED TO, PROCUREMENT OF SUBSTITUTE GOODS OR SERVICES;
% LOSS OF USE, DATA, OR PROFITS; OR BUSINESS INTERRUPTION) HOWEVER
% CAUSED AND ON ANY THEORY OF LIABILITY, WHETHER IN CONTRACT, STRICT
% LIABILITY, OR TORT (INCLUDING NEGLIGENCE OR OTHERWISE) ARISING IN
% ANY WAY OUT OF THE USE OF THIS SOFTWARE, EVEN IF ADVISED OF THE
% POSSIBILITY OF SUCH DAMAGE.
%%

%%
% This is the Eisvogel pandoc LaTeX template.
%
% For usage information and examples visit the official GitHub page:
% https://github.com/Wandmalfarbe/pandoc-latex-template
%%

% Options for packages loaded elsewhere
\PassOptionsToPackage{unicode}{hyperref}
\PassOptionsToPackage{hyphens}{url}
\PassOptionsToPackage{dvipsnames,svgnames*,x11names*,table}{xcolor}
%
\documentclass[
  paper=a4,
  ,captions=tableheading
]{scrartcl}
\usepackage{amsmath,amssymb}
\usepackage{lmodern}
\usepackage{setspace}
\setstretch{1.2}
\usepackage{amsmath}
\usepackage{ifxetex,ifluatex}
\ifnum 0\ifxetex 1\fi\ifluatex 1\fi=0 % if pdftex
  \usepackage[T1]{fontenc}
  \usepackage[utf8]{inputenc}
  \usepackage{textcomp} % provide euro and other symbols
  \usepackage{amssymb}
\else % if luatex or xetex
  \usepackage{unicode-math}
  \defaultfontfeatures{Scale=MatchLowercase}
  \defaultfontfeatures[\rmfamily]{Ligatures=TeX,Scale=1}
\fi
% Use upquote if available, for straight quotes in verbatim environments
\IfFileExists{upquote.sty}{\usepackage{upquote}}{}
\IfFileExists{microtype.sty}{% use microtype if available
  \usepackage[]{microtype}
  \UseMicrotypeSet[protrusion]{basicmath} % disable protrusion for tt fonts
}{}
\makeatletter
\@ifundefined{KOMAClassName}{% if non-KOMA class
  \IfFileExists{parskip.sty}{%
    \usepackage{parskip}
  }{% else
    \setlength{\parindent}{0pt}
    \setlength{\parskip}{6pt plus 2pt minus 1pt}}
}{% if KOMA class
  \KOMAoptions{parskip=half}}
\makeatother
\usepackage{xcolor}
\definecolor{default-linkcolor}{HTML}{A50000}
\definecolor{default-filecolor}{HTML}{A50000}
\definecolor{default-citecolor}{HTML}{4077C0}
\definecolor{default-urlcolor}{HTML}{4077C0}
\IfFileExists{xurl.sty}{\usepackage{xurl}}{} % add URL line breaks if available
\IfFileExists{bookmark.sty}{\usepackage{bookmark}}{\usepackage{hyperref}}
\hypersetup{
  pdftitle={Online Spiele},
  pdfauthor={Jonas Walkling},
  hidelinks,
  breaklinks=true,
  pdfcreator={LaTeX via pandoc with the Eisvogel template}}
\urlstyle{same} % disable monospaced font for URLs
\usepackage[margin=2.5cm,includehead=true,includefoot=true,centering,]{geometry}
\usepackage{listings}
\newcommand{\passthrough}[1]{#1}
\lstset{defaultdialect=[5.3]Lua}
\lstset{defaultdialect=[x86masm]Assembler}
% add backlinks to footnote references, cf. https://tex.stackexchange.com/questions/302266/make-footnote-clickable-both-ways
\usepackage{footnotebackref}
\setlength{\emergencystretch}{3em} % prevent overfull lines
\providecommand{\tightlist}{%
  \setlength{\itemsep}{0pt}\setlength{\parskip}{0pt}}
\setcounter{secnumdepth}{-\maxdimen} % remove section numbering

% Make use of float-package and set default placement for figures to H.
% The option H means 'PUT IT HERE' (as  opposed to the standard h option which means 'You may put it here if you like').
\usepackage{float}
\floatplacement{figure}{H}

\usepackage[T1]{fontenc}
\usepackage{tikz}
\usepackage{multicol}
\usepackage{hyperref}
\usepackage{svg}
\setcounter{secnumdepth}{3}
\setcounter{tocdepth}{1}
\renewcommand{\contentsname}{Spiele}
\ifluatex
  \usepackage{selnolig}  % disable illegal ligatures
\fi

\title{Online Spiele}
\author{Jonas Walkling}
\date{}



%%
%% added
%%

%
% language specification
%
% If no language is specified, use English as the default main document language.
%

\ifnum 0\ifxetex 1\fi\ifluatex 1\fi=0 % if pdftex
  \usepackage[shorthands=off,main=english]{babel}
\else
    % Workaround for bug in Polyglossia that breaks `\familydefault` when `\setmainlanguage` is used.
  % See https://github.com/Wandmalfarbe/pandoc-latex-template/issues/8
  % See https://github.com/reutenauer/polyglossia/issues/186
  % See https://github.com/reutenauer/polyglossia/issues/127
  \renewcommand*\familydefault{\sfdefault}
    % load polyglossia as late as possible as it *could* call bidi if RTL lang (e.g. Hebrew or Arabic)
  \usepackage{polyglossia}
  \setmainlanguage[]{english}
\fi



%
% for the background color of the title page
%

%
% break urls
%
\PassOptionsToPackage{hyphens}{url}

%
% When using babel or polyglossia with biblatex, loading csquotes is recommended
% to ensure that quoted texts are typeset according to the rules of your main language.
%
\usepackage{csquotes}

%
% captions
%
\definecolor{caption-color}{HTML}{777777}
\usepackage[font={stretch=1.2}, textfont={color=caption-color}, position=top, skip=4mm, labelfont=bf, singlelinecheck=false, justification=raggedright]{caption}
\setcapindent{0em}

%
% blockquote
%
\definecolor{blockquote-border}{RGB}{221,221,221}
\definecolor{blockquote-text}{RGB}{119,119,119}
\usepackage{mdframed}
\newmdenv[rightline=false,bottomline=false,topline=false,linewidth=3pt,linecolor=blockquote-border,skipabove=\parskip]{customblockquote}
\renewenvironment{quote}{\begin{customblockquote}\list{}{\rightmargin=0em\leftmargin=0em}%
\item\relax\color{blockquote-text}\ignorespaces}{\unskip\unskip\endlist\end{customblockquote}}

%
% Source Sans Pro as the de­fault font fam­ily
% Source Code Pro for monospace text
%
% 'default' option sets the default
% font family to Source Sans Pro, not \sfdefault.
%
\ifnum 0\ifxetex 1\fi\ifluatex 1\fi=0 % if pdftex
    \usepackage[default]{sourcesanspro}
  \usepackage{sourcecodepro}
  \else % if not pdftex
    \usepackage[default]{sourcesanspro}
  \usepackage{sourcecodepro}

  % XeLaTeX specific adjustments for straight quotes: https://tex.stackexchange.com/a/354887
  % This issue is already fixed (see https://github.com/silkeh/latex-sourcecodepro/pull/5) but the
  % fix is still unreleased.
  % TODO: Remove this workaround when the new version of sourcecodepro is released on CTAN.
  \ifxetex
    \makeatletter
    \defaultfontfeatures[\ttfamily]
      { Numbers   = \sourcecodepro@figurestyle,
        Scale     = \SourceCodePro@scale,
        Extension = .otf }
    \setmonofont
      [ UprightFont    = *-\sourcecodepro@regstyle,
        ItalicFont     = *-\sourcecodepro@regstyle It,
        BoldFont       = *-\sourcecodepro@boldstyle,
        BoldItalicFont = *-\sourcecodepro@boldstyle It ]
      {SourceCodePro}
    \makeatother
  \fi
  \fi

%
% heading color
%
\definecolor{heading-color}{RGB}{40,40,40}
\addtokomafont{section}{\color{heading-color}}
% When using the classes report, scrreprt, book,
% scrbook or memoir, uncomment the following line.
%\addtokomafont{chapter}{\color{heading-color}}

%
% variables for title, author and date
%
\usepackage{titling}
\title{Online Spiele}
\author{Jonas Walkling}
\date{}

%
% tables
%

%
% remove paragraph indention
%
\setlength{\parindent}{0pt}
\setlength{\parskip}{6pt plus 2pt minus 1pt}
\setlength{\emergencystretch}{3em}  % prevent overfull lines

%
%
% Listings
%
%


%
% general listing colors
%
\definecolor{listing-background}{HTML}{F7F7F7}
\definecolor{listing-rule}{HTML}{B3B2B3}
\definecolor{listing-numbers}{HTML}{B3B2B3}
\definecolor{listing-text-color}{HTML}{000000}
\definecolor{listing-keyword}{HTML}{435489}
\definecolor{listing-keyword-2}{HTML}{1284CA} % additional keywords
\definecolor{listing-keyword-3}{HTML}{9137CB} % additional keywords
\definecolor{listing-identifier}{HTML}{435489}
\definecolor{listing-string}{HTML}{00999A}
\definecolor{listing-comment}{HTML}{8E8E8E}

\lstdefinestyle{eisvogel_listing_style}{
  language         = java,
  numbers          = left,
  xleftmargin      = 2.7em,
  framexleftmargin = 2.5em,
  backgroundcolor  = \color{listing-background},
  basicstyle       = \color{listing-text-color}\linespread{1.0}\small\ttfamily{},
  breaklines       = true,
  frame            = single,
  framesep         = 0.19em,
  rulecolor        = \color{listing-rule},
  frameround       = ffff,
  tabsize          = 4,
  numberstyle      = \color{listing-numbers},
  aboveskip        = 1.0em,
  belowskip        = 0.1em,
  abovecaptionskip = 0em,
  belowcaptionskip = 1.0em,
  keywordstyle     = {\color{listing-keyword}\bfseries},
  keywordstyle     = {[2]\color{listing-keyword-2}\bfseries},
  keywordstyle     = {[3]\color{listing-keyword-3}\bfseries\itshape},
  sensitive        = true,
  identifierstyle  = \color{listing-identifier},
  commentstyle     = \color{listing-comment},
  stringstyle      = \color{listing-string},
  showstringspaces = false,
  escapeinside     = {/*@}{@*/}, % Allow LaTeX inside these special comments
  literate         =
  {á}{{\'a}}1 {é}{{\'e}}1 {í}{{\'i}}1 {ó}{{\'o}}1 {ú}{{\'u}}1
  {Á}{{\'A}}1 {É}{{\'E}}1 {Í}{{\'I}}1 {Ó}{{\'O}}1 {Ú}{{\'U}}1
  {à}{{\`a}}1 {è}{{\'e}}1 {ì}{{\`i}}1 {ò}{{\`o}}1 {ù}{{\`u}}1
  {À}{{\`A}}1 {È}{{\'E}}1 {Ì}{{\`I}}1 {Ò}{{\`O}}1 {Ù}{{\`U}}1
  {ä}{{\"a}}1 {ë}{{\"e}}1 {ï}{{\"i}}1 {ö}{{\"o}}1 {ü}{{\"u}}1
  {Ä}{{\"A}}1 {Ë}{{\"E}}1 {Ï}{{\"I}}1 {Ö}{{\"O}}1 {Ü}{{\"U}}1
  {â}{{\^a}}1 {ê}{{\^e}}1 {î}{{\^i}}1 {ô}{{\^o}}1 {û}{{\^u}}1
  {Â}{{\^A}}1 {Ê}{{\^E}}1 {Î}{{\^I}}1 {Ô}{{\^O}}1 {Û}{{\^U}}1
  {œ}{{\oe}}1 {Œ}{{\OE}}1 {æ}{{\ae}}1 {Æ}{{\AE}}1 {ß}{{\ss}}1
  {ç}{{\c c}}1 {Ç}{{\c C}}1 {ø}{{\o}}1 {å}{{\r a}}1 {Å}{{\r A}}1
  {€}{{\EUR}}1 {£}{{\pounds}}1 {«}{{\guillemotleft}}1
  {»}{{\guillemotright}}1 {ñ}{{\~n}}1 {Ñ}{{\~N}}1 {¿}{{?`}}1
  {…}{{\ldots}}1 {≥}{{>=}}1 {≤}{{<=}}1 {„}{{\glqq}}1 {“}{{\grqq}}1
  {”}{{''}}1
}
\lstset{style=eisvogel_listing_style}

%
% Java (Java SE 12, 2019-06-22)
%
\lstdefinelanguage{Java}{
  morekeywords={
    % normal keywords (without data types)
    abstract,assert,break,case,catch,class,continue,default,
    do,else,enum,exports,extends,final,finally,for,if,implements,
    import,instanceof,interface,module,native,new,package,private,
    protected,public,requires,return,static,strictfp,super,switch,
    synchronized,this,throw,throws,transient,try,volatile,while,
    % var is an identifier
    var
  },
  morekeywords={[2] % data types
    % primitive data types
    boolean,byte,char,double,float,int,long,short,
    % String
    String,
    % primitive wrapper types
    Boolean,Byte,Character,Double,Float,Integer,Long,Short
    % number types
    Number,AtomicInteger,AtomicLong,BigDecimal,BigInteger,DoubleAccumulator,DoubleAdder,LongAccumulator,LongAdder,Short,
    % other
    Object,Void,void
  },
  morekeywords={[3] % literals
    % reserved words for literal values
    null,true,false,
  },
  sensitive,
  morecomment  = [l]//,
  morecomment  = [s]{/*}{*/},
  morecomment  = [s]{/**}{*/},
  morestring   = [b]",
  morestring   = [b]',
}

\lstdefinelanguage{XML}{
  morestring      = [b]",
  moredelim       = [s][\bfseries\color{listing-keyword}]{<}{\ },
  moredelim       = [s][\bfseries\color{listing-keyword}]{</}{>},
  moredelim       = [l][\bfseries\color{listing-keyword}]{/>},
  moredelim       = [l][\bfseries\color{listing-keyword}]{>},
  morecomment     = [s]{<?}{?>},
  morecomment     = [s]{<!--}{-->},
  commentstyle    = \color{listing-comment},
  stringstyle     = \color{listing-string},
  identifierstyle = \color{listing-identifier}
}

%
% header and footer
%

%%
%% end added
%%

\begin{document}

%%
%% begin titlepage
%%

%%
%% end titlepage
%%



\maketitle
\tableofcontents

\section*{URL-Parameter}

\begin{multicols}{2}


Ein URL Parameter gibt Informationen an die aufgerufene Internetseite weiter. Sie stehen nach der Adresse -- eingeleitet durch ein ?.
Ein Wert wird dann mit dem Namen des Wertes (Key), einem '=' und der zugehörigen Zahl, nach dem Muster key=zahl angegeben. Die Spiele sind so konzipiert, dass alle Werte einen Standard besitzen (als Default angegeben). Wenn kein Wert über die URL eingegeben wird, verwendet das Programm den Defaultwert.
Die Range ist ein Angabe in Welchem der Wert liegen muss bzw. in welchem Bereich, das Spiel immer noch Spielbar ist. Bei $\in \mathbb{N}$ muss es eine Ganzzahl sein, bei $\in \mathbb{Q}$ kann es eine Kommazahl sein. Falls Wörter (strings) oder bestimmte Zahlen verlanget werden ist dies in den Docs der einzelnen Spiele angegeben.

\end{multicols}

\hypertarget{einmaleins}{%
\section{Einmaleins}\label{einmaleins}}

\hypertarget{reihe}{%
\paragraph{reihe}\label{reihe}}

Einmaleins Reihe. Das Spiel kann auch größere Reihen als 1-10. Zahlen
bis zu vier Stellen werden korrekt angezeigt.

\begin{itemize}
\tightlist
\item
  Default: 7
\item
  Range: 1-999 \(\in \mathbb{N}\)
\end{itemize}

\hypertarget{dauer}{%
\paragraph{dauer}\label{dauer}}

Spieldauer in Sekunden

\begin{itemize}
\tightlist
\item
  Default: 45
\item
  Range: 0 - \(\infty \in \mathbb{N}\)
\end{itemize}

\hypertarget{prozent}{%
\paragraph{prozent}\label{prozent}}

Anteil der Zahnräder die von allen richtigen Zahnrädern gefangen werden
müssen damit man das Spiel gewinnt.

\begin{itemize}
\tightlist
\item
  Default: 0.75
\item
  Range: 0 - 1 \(\in \mathbb{Q}\)
\end{itemize}

\hypertarget{speed}{%
\paragraph{speed}\label{speed}}

Laufgeschwindigkeit des Roboters

\begin{itemize}
\tightlist
\item
  Default: 30
\item
  Range: 1 - \(\infty \in \mathbb{N}\)
\end{itemize}

\hypertarget{schwerkraft}{%
\paragraph{schwerkraft}\label{schwerkraft}}

Fallgeschwindigkeit der Zahnräder

\begin{itemize}
\tightlist
\item
  Default: 3
\item
  Range: 1 - 9 \(\in \mathbb{N}\)
\end{itemize}

\hypertarget{interval}{%
\paragraph{interval}\label{interval}}

Zeitspanne zwischen dem Erscheinen neuer Zahnräder

\begin{itemize}
\tightlist
\item
  Default: 2
\item
  Range: 0.1 - \(\infty \in \mathbb{Q}\)
\end{itemize}

\hypertarget{max}{%
\paragraph{max}\label{max}}

Faktor für die maximale Anzahl an Zahnrädern, die gleichzeitig auf dem
Bildschirm zu sehen sind. max ist ein Multiplikator für die aus dem
Quotient der Breite des Browserfensters und der Breite des Roboters
berechneten Maximalanzahl. Wenn dieser Wert eher hoch ist (
\textgreater{} 5 ) ist für die Anzahl an Zahnrädern das Interval
zwischen dem Erscheinen neuer Zahnräder ausschlaggebend.

\begin{itemize}
\tightlist
\item
  Default: 2
\item
  Range: 0.1 - \(\infty \in \mathbb{Q}\)
\end{itemize}

\hypertarget{fps}{%
\paragraph{fps}\label{fps}}

Spielgeschwindigkeit\\
Gibt an wie oft das Bild pro Sekunde neu berechnet wird. Ein größerer
Wert bedeutet eine schnellere Lauf- und Fallgeschwindigkeit.

\begin{itemize}
\tightlist
\item
  Default: 30
\item
  Range: 10 - 120 \(\in \mathbb{N}\)
\end{itemize}

\hypertarget{breite}{%
\paragraph{breite}\label{breite}}

Breite des Roboters in Pixeln

\begin{itemize}
\tightlist
\item
  Default: 200
\item
  Range: 75 - 400 \(\in \mathbb{N}\)
\end{itemize}

\hypertarget{zr_scale}{%
\paragraph{zr\_scale}\label{zr_scale}}

Skalierung der Zahnradgröße im Verhältnis zum Roboter. Bei einem Wert
von 1 hat das Zahnrad die 0.4-fache Breite des Roboters.

\begin{itemize}
\tightlist
\item
  Default: 1
\item
  0.5 - 3 \(\in \mathbb{Q}\)
\end{itemize}

\hypertarget{anlaute}{%
\section{Anlaute}\label{anlaute}}

\hypertarget{l}{%
\paragraph{l}\label{l}}

Legt das Level fest\\

\begin{itemize}
\tightlist
\item
  Default: 4
\item
  Möglich: 1 - 4 \(\in \mathbb{N}\)

  \begin{itemize}
  \tightlist
  \item
    Level 1: A, E, I, O, U, M, L, S
  \item
    Level 2: Level 1 + W, R, F, N, T, Au, Ei
  \item
    Level 3: Level 2 + H, D, Sch, K, Z, P, G, J, EU
  \item
    Level 4: alle Anlaute
  \end{itemize}
\end{itemize}

\hypertarget{c}{%
\paragraph{c}\label{c}}

Legt fest wie sehr sich die Karten am Anfang abstoßen.

\begin{itemize}
\tightlist
\item
  Default: 2
\item
  Möglich: 1 - 4 \(\in \mathbb{Q}\)

  \item[]
  \item[]
  \item[]
\end{itemize}

\hypertarget{mengen}{%
\section{Mengen}\label{mengen}}

\hypertarget{set}{%
\paragraph{set}\label{set}}

Legt die verwendeten Bilder fest

\begin{itemize}
\tightlist
\item
  Default: augen
\item
  Möglich: augen, finger, punkte, striche
\end{itemize}

\hypertarget{t}{%
\paragraph{t}\label{t}}

Anzeigezeit in Millisekunden

\begin{itemize}
\tightlist
\item
  Default: 7
\item
  Range: 100 - 1500 \(\in \mathbb{N}\)
\end{itemize}

\hypertarget{s}{%
\paragraph{s}\label{s}}

Verkürzung der Anzeigezeit pro Runde in Prozent. Zum Beispiel wird für
s=2 die Anzeigezeit pro runde um 2\% verringert.

\begin{itemize}
\tightlist
\item
  Default: 0
\item
  Range: 1 - 10 \(\in \mathbb{N}\)
\end{itemize}

\hypertarget{r}{%
\paragraph{r}\label{r}}

Runden pro Durchgang

\begin{itemize}
\tightlist
\item
  Default: 20
\item
  Range: 5 - 30 \(\in \mathbb{N}\)
\end{itemize}

\hypertarget{c-1}{%
\paragraph{c}\label{c-1}}

Wenn c gesetzt wird werden keine Hintergrundfarben für die Buttons
angezeigt.

\begin{itemize}
\tightlist
\item
  Default: An
\item
  Range: egal (ASCII-Zeichen)
\end{itemize}

\hypertarget{p}{%
\paragraph{p}\label{p}}

Wenn p gesetzt wird, wird das Endergebnis in Prozent, statt als x von y
angezeigt.

\begin{itemize}
\tightlist
\item
  Default: x von y
\item
  Range: egal (ASCII-Zeichen)
\end{itemize}

\hypertarget{inlaute}{%
\section{Inlaute}\label{inlaute}}

\hypertarget{beispiel}{%
\paragraph{Beispiel:}\label{beispiel}}

\url{laagbergschule.de/lernspiele/deutsch/inlaute?s=m\&n=8}

\hypertarget{s-1}{%
\paragraph{s}\label{s-1}}

Legt den Buchstaben fest

\begin{itemize}
\tightlist
\item
  Default: -
\item
  Möglich: m \bigskip
\end{itemize}

\hypertarget{n}{%
\paragraph{n}\label{n}}

Legt die Anzal der Karten fest. Wenn die Anzahl der Karten kleiner ist,
als die verfügbaren Wörter, werden n zufällige Wörter ausgewählt.

\begin{itemize}
\tightlist
\item
  Default: 10
\item
  Möglich: 3 - Anzahl der Karten für einen Buchstaben \(\in \mathbb{N}\)
\item
\end{itemize}

\hypertarget{immerzehn}{%
\section{Immerzehn}\label{immerzehn}}

\href{https://www.laagbergschule.de/lernspiele/mathe/immerzehn?d=4\&t=40}{laagbergschule.de/lernspiele/mathe/immerzehn?d=4\&t=40}

\hypertarget{d}{%
\paragraph{d}\label{d}}

Legt die größe des Spielfeldes fest, z.B. d=4 gibt ein 4x4 Feld. Der
Wert muss gerade sein, da das Spiel sonst nicht aufgehen kann.

\begin{itemize}
\tightlist
\item
  Default: 4 x 6
\item
  Möglich: 2,4,6,8,10 \(\in \mathbb{N}\)
\end{itemize}

\hypertarget{c-2}{%
\paragraph{c}\label{c-2}}

Legt die Anzahl der Spalten fest. Muss mit den Reihen multipliziert eine
gerade Zahl ergeben.

\begin{itemize}
\tightlist
\item
  Default: 4
\item
  Möglich: 2 - 10 \(\in \mathbb{N}\)
\end{itemize}

\hypertarget{r-1}{%
\paragraph{r}\label{r-1}}

Legt die Anzahl der Reihen fest. Muss mit den Spalten multiplizert eine
gerade Zahl ergeben.

\begin{itemize}
\tightlist
\item
  Default: 6
\item
  Möglich: 2 - 10 \(\in \mathbb{N}\)
\end{itemize}

\hypertarget{t-1}{%
\paragraph{t}\label{t-1}}

Legt die zeit pro Runde fest.

\begin{itemize}
\tightlist
\item
  Default: 60
\item
  Möglich: 20-120 \(\in \mathbb{N}\)
\end{itemize}

\hypertarget{g}{%
\paragraph{g}\label{g}}

Legt die Zahl fest, die das Ergenis sein soll.

\begin{itemize}
\tightlist
\item
  Default: 10
\item
  Möglich: 2-16 \(\in \mathbb{N}\)
\end{itemize}

\end{document}
