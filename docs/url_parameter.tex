\documentclass[a4paper]{article}

\usepackage{amsmath, amssymb, amsthm}
\usepackage[T1]{fontenc}
\usepackage{multicol}
\usepackage{CrimsonPro}
\usepackage[utf8]{inputenc}
\usepackage{geometry}
\usepackage{hyperref}
\usepackage{multicol}
\title{URL-Parameter}

\begin{document}

\maketitle

\begin{multicols}{2}
Ein URL Parameter gibt Informationen an die aufgerufene Internetseite weiter. Sie stehen nach der Adresse -- eingeleitet durch ein ?.
Ein Wert wird dann mit dem Namen des Wertes (Key), einem '=' und der zugehörigen Zahl, nach dem Muster key=zahl angegeben. Die Spiele sind so konzipiert, dass alle Werte einen Standard besitzen (als Default angegeben). Wenn kein Wert über die URL eingegeben wird, verwendet das Programm den Defaultwert.
Die Range ist ein Angabe in Welchem der Wert liegen muss bzw. in welchem Bereich, das Spiel immer noch Spielbar ist. Bei $\in \mathbb{N} $ muss es eine Ganzzahl sein, bei $\in \mathbb{Q} $ kann es eine Kommazahl sein. Falls Wörter (strings) oder bestimmte Zahlen verlanget werden ist dies in den Docs der einzelnen Spiele angegeben.
\end{multicols}

\subsection*{Drei Beispiele:}
\texttt{\href{onecalfman.github.io/einmaleins?reihe=5}{onecalfman.github.io/einmaleins?reihe=5}} \\
öffnet das Spiel mit der 5er Reihe voreingestellt.
Weiter Optionen werden über ein \& angehängt. \\\\

\texttt{\href{onecalfman.github.io/einmaleins?reihe=8\&dauer=60\&max=5}{onecalfman.github.io/einmaleins?reihe=8\&dauer=60\&max=5}} \\
öffnet das Spiel mit der 8er Reihe, einer Spieldauer von 60 Sekunden und maximal 5 Zahnrädern die Gleichzeitig zu sehen sind. Die möglichen keys sind unten gelistet. \\\\

\texttt{\href{onecalfman.github.io/mengen?set=finger\&t=500\&s=5\&p=1\&r=10}{onecalfman.github.io/mengen?set=finger\&t=500\&s=5\&p=1\&r=10}} \\
öffnet das Mengenspiel mit einer Anzeigezeit von 500 ms, einer Verkürzung von 5\% pro Runde dem Endergebnis als Prozent angegeben und 10 Runden pro Durchgang.

\end{document}


